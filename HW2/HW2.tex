% Options for packages loaded elsewhere
\PassOptionsToPackage{unicode}{hyperref}
\PassOptionsToPackage{hyphens}{url}
%
\documentclass[
]{article}
\usepackage{lmodern}
\usepackage{amssymb,amsmath}
\usepackage{ifxetex,ifluatex}
\ifnum 0\ifxetex 1\fi\ifluatex 1\fi=0 % if pdftex
  \usepackage[T1]{fontenc}
  \usepackage[utf8]{inputenc}
  \usepackage{textcomp} % provide euro and other symbols
\else % if luatex or xetex
  \usepackage{unicode-math}
  \defaultfontfeatures{Scale=MatchLowercase}
  \defaultfontfeatures[\rmfamily]{Ligatures=TeX,Scale=1}
\fi
% Use upquote if available, for straight quotes in verbatim environments
\IfFileExists{upquote.sty}{\usepackage{upquote}}{}
\IfFileExists{microtype.sty}{% use microtype if available
  \usepackage[]{microtype}
  \UseMicrotypeSet[protrusion]{basicmath} % disable protrusion for tt fonts
}{}
\makeatletter
\@ifundefined{KOMAClassName}{% if non-KOMA class
  \IfFileExists{parskip.sty}{%
    \usepackage{parskip}
  }{% else
    \setlength{\parindent}{0pt}
    \setlength{\parskip}{6pt plus 2pt minus 1pt}}
}{% if KOMA class
  \KOMAoptions{parskip=half}}
\makeatother
\usepackage{xcolor}
\IfFileExists{xurl.sty}{\usepackage{xurl}}{} % add URL line breaks if available
\IfFileExists{bookmark.sty}{\usepackage{bookmark}}{\usepackage{hyperref}}
\hypersetup{
  pdftitle={HW2},
  pdfauthor={Nikhil Gopal},
  hidelinks,
  pdfcreator={LaTeX via pandoc}}
\urlstyle{same} % disable monospaced font for URLs
\usepackage[margin=1in]{geometry}
\usepackage{color}
\usepackage{fancyvrb}
\newcommand{\VerbBar}{|}
\newcommand{\VERB}{\Verb[commandchars=\\\{\}]}
\DefineVerbatimEnvironment{Highlighting}{Verbatim}{commandchars=\\\{\}}
% Add ',fontsize=\small' for more characters per line
\usepackage{framed}
\definecolor{shadecolor}{RGB}{248,248,248}
\newenvironment{Shaded}{\begin{snugshade}}{\end{snugshade}}
\newcommand{\AlertTok}[1]{\textcolor[rgb]{0.94,0.16,0.16}{#1}}
\newcommand{\AnnotationTok}[1]{\textcolor[rgb]{0.56,0.35,0.01}{\textbf{\textit{#1}}}}
\newcommand{\AttributeTok}[1]{\textcolor[rgb]{0.77,0.63,0.00}{#1}}
\newcommand{\BaseNTok}[1]{\textcolor[rgb]{0.00,0.00,0.81}{#1}}
\newcommand{\BuiltInTok}[1]{#1}
\newcommand{\CharTok}[1]{\textcolor[rgb]{0.31,0.60,0.02}{#1}}
\newcommand{\CommentTok}[1]{\textcolor[rgb]{0.56,0.35,0.01}{\textit{#1}}}
\newcommand{\CommentVarTok}[1]{\textcolor[rgb]{0.56,0.35,0.01}{\textbf{\textit{#1}}}}
\newcommand{\ConstantTok}[1]{\textcolor[rgb]{0.00,0.00,0.00}{#1}}
\newcommand{\ControlFlowTok}[1]{\textcolor[rgb]{0.13,0.29,0.53}{\textbf{#1}}}
\newcommand{\DataTypeTok}[1]{\textcolor[rgb]{0.13,0.29,0.53}{#1}}
\newcommand{\DecValTok}[1]{\textcolor[rgb]{0.00,0.00,0.81}{#1}}
\newcommand{\DocumentationTok}[1]{\textcolor[rgb]{0.56,0.35,0.01}{\textbf{\textit{#1}}}}
\newcommand{\ErrorTok}[1]{\textcolor[rgb]{0.64,0.00,0.00}{\textbf{#1}}}
\newcommand{\ExtensionTok}[1]{#1}
\newcommand{\FloatTok}[1]{\textcolor[rgb]{0.00,0.00,0.81}{#1}}
\newcommand{\FunctionTok}[1]{\textcolor[rgb]{0.00,0.00,0.00}{#1}}
\newcommand{\ImportTok}[1]{#1}
\newcommand{\InformationTok}[1]{\textcolor[rgb]{0.56,0.35,0.01}{\textbf{\textit{#1}}}}
\newcommand{\KeywordTok}[1]{\textcolor[rgb]{0.13,0.29,0.53}{\textbf{#1}}}
\newcommand{\NormalTok}[1]{#1}
\newcommand{\OperatorTok}[1]{\textcolor[rgb]{0.81,0.36,0.00}{\textbf{#1}}}
\newcommand{\OtherTok}[1]{\textcolor[rgb]{0.56,0.35,0.01}{#1}}
\newcommand{\PreprocessorTok}[1]{\textcolor[rgb]{0.56,0.35,0.01}{\textit{#1}}}
\newcommand{\RegionMarkerTok}[1]{#1}
\newcommand{\SpecialCharTok}[1]{\textcolor[rgb]{0.00,0.00,0.00}{#1}}
\newcommand{\SpecialStringTok}[1]{\textcolor[rgb]{0.31,0.60,0.02}{#1}}
\newcommand{\StringTok}[1]{\textcolor[rgb]{0.31,0.60,0.02}{#1}}
\newcommand{\VariableTok}[1]{\textcolor[rgb]{0.00,0.00,0.00}{#1}}
\newcommand{\VerbatimStringTok}[1]{\textcolor[rgb]{0.31,0.60,0.02}{#1}}
\newcommand{\WarningTok}[1]{\textcolor[rgb]{0.56,0.35,0.01}{\textbf{\textit{#1}}}}
\usepackage{graphicx,grffile}
\makeatletter
\def\maxwidth{\ifdim\Gin@nat@width>\linewidth\linewidth\else\Gin@nat@width\fi}
\def\maxheight{\ifdim\Gin@nat@height>\textheight\textheight\else\Gin@nat@height\fi}
\makeatother
% Scale images if necessary, so that they will not overflow the page
% margins by default, and it is still possible to overwrite the defaults
% using explicit options in \includegraphics[width, height, ...]{}
\setkeys{Gin}{width=\maxwidth,height=\maxheight,keepaspectratio}
% Set default figure placement to htbp
\makeatletter
\def\fps@figure{htbp}
\makeatother
\setlength{\emergencystretch}{3em} % prevent overfull lines
\providecommand{\tightlist}{%
  \setlength{\itemsep}{0pt}\setlength{\parskip}{0pt}}
\setcounter{secnumdepth}{-\maxdimen} % remove section numbering

\title{HW2}
\author{Nikhil Gopal}
\date{2/8/2021}

\begin{document}
\maketitle

\textbf{Question 1}

Estimate for odds ratio is about 220:

\begin{Shaded}
\begin{Highlighting}[]
\KeywordTok{setwd}\NormalTok{(}\StringTok{"C:/Users/d/Google Drive/Notability/Categorical Data/psets/HW2"}\NormalTok{)}

\CommentTok{#Odds ratio: }

\NormalTok{(}\DecValTok{802}\OperatorTok{*}\DecValTok{494}\NormalTok{) }\OperatorTok{/}\StringTok{ }\NormalTok{(}\DecValTok{34}\OperatorTok{*}\DecValTok{53}\NormalTok{)}
\end{Highlighting}
\end{Shaded}

\begin{verbatim}
## [1] 219.8602
\end{verbatim}

This means that someone who voted for Obama in 2008 was about 220 times
more likely to vote for Obama in 2012 than someone who voted for McCain.

\#95\% CI/OR

\begin{Shaded}
\begin{Highlighting}[]
\KeywordTok{library}\NormalTok{(epitools)}
\end{Highlighting}
\end{Shaded}

\begin{verbatim}
## Warning: package 'epitools' was built under R version 4.0.3
\end{verbatim}

\begin{Shaded}
\begin{Highlighting}[]
\NormalTok{table<-}\KeywordTok{matrix}\NormalTok{(}\KeywordTok{c}\NormalTok{(}\DecValTok{802}\NormalTok{,}\DecValTok{34}\NormalTok{,}\DecValTok{53}\NormalTok{,}\DecValTok{494}\NormalTok{),}\DataTypeTok{nrow =} \DecValTok{2}\NormalTok{, }\DataTypeTok{ncol =} \DecValTok{2}\NormalTok{)}

\KeywordTok{rownames}\NormalTok{(table) <-}\StringTok{ }\KeywordTok{c}\NormalTok{(}\StringTok{"Obama"}\NormalTok{, }\StringTok{"McCain"}\NormalTok{)}
\KeywordTok{colnames}\NormalTok{(table) <-}\StringTok{ }\KeywordTok{c}\NormalTok{(}\StringTok{"Obama"}\NormalTok{, }\StringTok{"Romney"}\NormalTok{)}

\NormalTok{table}
\end{Highlighting}
\end{Shaded}

\begin{verbatim}
##        Obama Romney
## Obama    802     53
## McCain    34    494
\end{verbatim}

\begin{Shaded}
\begin{Highlighting}[]
\KeywordTok{oddsratio.wald}\NormalTok{(table)}
\end{Highlighting}
\end{Shaded}

\begin{verbatim}
## $data
##        Obama Romney Total
## Obama    802     53   855
## McCain    34    494   528
## Total    836    547  1383
## 
## $measure
##                         NA
## odds ratio with 95% C.I. estimate    lower    upper
##                   Obama    1.0000       NA       NA
##                   McCain 219.8602 140.8909 343.0917
## 
## $p.value
##          NA
## two-sided midp.exact  fisher.exact    chi.square
##    Obama          NA            NA            NA
##    McCain          0 1.673741e-263 1.327956e-228
## 
## $correction
## [1] FALSE
## 
## attr(,"method")
## [1] "Unconditional MLE & normal approximation (Wald) CI"
\end{verbatim}

The 95\% CI for the population odds-ratio was 140.89-343.0917. We are
95\% confident that the true population odds ratio lies within the
interval.

\textbf{Question 2}

\begin{Shaded}
\begin{Highlighting}[]
\KeywordTok{library}\NormalTok{(DescTools)}
\end{Highlighting}
\end{Shaded}

\begin{verbatim}
## Warning: package 'DescTools' was built under R version 4.0.3
\end{verbatim}

\begin{Shaded}
\begin{Highlighting}[]
\NormalTok{table2<-}\KeywordTok{matrix}\NormalTok{(}\KeywordTok{c}\NormalTok{(}\DecValTok{871}\NormalTok{,}\DecValTok{347}\NormalTok{,}\DecValTok{821}\NormalTok{,}\DecValTok{42}\NormalTok{,}\DecValTok{336}\NormalTok{,}\DecValTok{83}\NormalTok{),}\DataTypeTok{nrow =} \DecValTok{2}\NormalTok{, }\DataTypeTok{ncol =} \DecValTok{3}\NormalTok{)}

\KeywordTok{rownames}\NormalTok{(table2) <-}\StringTok{ }\KeywordTok{c}\NormalTok{(}\StringTok{"White"}\NormalTok{, }\StringTok{"Black"}\NormalTok{)}
\KeywordTok{colnames}\NormalTok{(table2) <-}\StringTok{ }\KeywordTok{c}\NormalTok{(}\StringTok{"Democrat"}\NormalTok{, }\StringTok{"Republican"}\NormalTok{, }\StringTok{"Independent"}\NormalTok{)}

\CommentTok{#Chi Square}
\KeywordTok{chisq.test}\NormalTok{(table2)}
\end{Highlighting}
\end{Shaded}

\begin{verbatim}
## 
##  Pearson's Chi-squared test
## 
## data:  table2
## X-squared = 184.32, df = 2, p-value < 2.2e-16
\end{verbatim}

\begin{Shaded}
\begin{Highlighting}[]
\CommentTok{#Log-Likelihood}
\NormalTok{DescTools}\OperatorTok{::}\KeywordTok{GTest}\NormalTok{(table2,}
                 \DataTypeTok{correct =} \StringTok{"none"}\NormalTok{)}
\end{Highlighting}
\end{Shaded}

\begin{verbatim}
## 
##  Log likelihood ratio (G-test) test of independence without correction
## 
## data:  table2
## G = 213.9, X-squared df = 2, p-value < 2.2e-16
\end{verbatim}

Both the Chi-square and Log-likelihood tests returned p values that were
less than 0.01. We can thus conclude with statistical significance that
there political affiliation and race are independent

Residuals:

\begin{Shaded}
\begin{Highlighting}[]
\KeywordTok{chisq.test}\NormalTok{(table2)}\OperatorTok{$}\NormalTok{stdres}
\end{Highlighting}
\end{Shaded}

\begin{verbatim}
##        Democrat Republican Independent
## White -11.96679   12.99946  -0.5326281
## Black  11.96679  -12.99946   0.5326281
\end{verbatim}

Standardized residuals give the predicted-expected frequencies in a
contigency table divided by the standard error. The residuals are
standardized because dividing by standard error means that the residuals
are normally distributed. For democrats, whites returned a residual of
-11.966 and for Republicans blacks returned a residual of -12.999. Since
these residuals are standardized, this provides extremely strong
evidence that more people would be in these cells if party ID were
independent of race, which helps provide more evidence against our null
hypothesis that party is not independent of race.

\textbf{Question 3}

It is not valid to apply the Chi-square test for this table, since the
same person can be in multiple columns. Participants were instructed to
select however many columns they thought were true, which violates the
independence assumption of the chi-square test.

We are able to cross classify because we know how many people were
sampled originally. subtracting the observed values from 100 will give
us the number of people of each gender who said no to that variable
being responsible for an increase in teenage crime.

table for part c:

\begin{Shaded}
\begin{Highlighting}[]
\CommentTok{#contingency table for factor A}
\NormalTok{table3 <-}\StringTok{ }\KeywordTok{matrix}\NormalTok{(}\KeywordTok{c}\NormalTok{(}\DecValTok{60}\NormalTok{,}\DecValTok{75}\NormalTok{,}\DecValTok{100-60}\NormalTok{,}\DecValTok{100-75}\NormalTok{,}\DecValTok{100}\NormalTok{,}\DecValTok{100}\NormalTok{), }\DataTypeTok{nrow =} \DecValTok{2}\NormalTok{, }\DataTypeTok{ncol =} \DecValTok{3}\NormalTok{) }
\KeywordTok{rownames}\NormalTok{(table3) <-}\StringTok{ }\KeywordTok{c}\NormalTok{(}\StringTok{"Men"}\NormalTok{, }\StringTok{"Women"}\NormalTok{)}
\KeywordTok{colnames}\NormalTok{(table3) <-}\StringTok{ }\KeywordTok{c}\NormalTok{(}\StringTok{"Y"}\NormalTok{, }\StringTok{"N"}\NormalTok{, }\StringTok{"Total"}\NormalTok{)}
\NormalTok{table3}
\end{Highlighting}
\end{Shaded}

\begin{verbatim}
##        Y  N Total
## Men   60 40   100
## Women 75 25   100
\end{verbatim}

d:

Odds ratio =

\[=\frac{n_{11}n_{22}}{n_{12}n_{21}} = \frac{\frac{n_{1+}n_{+1}}{n}\frac{n_{2+}n_{+2}}{n}}{\frac{n_{1+}n_{+2}}{n}\frac{n_{2+}n_{+1}}{n}} = \frac{{\hat{\mu}}_{11}{\hat{\mu}}_{22}}{{\hat{\mu}}_{12}{\hat{\mu}}_{21}} = 1\]

Therefore, \({\hat{\mu}}_{ij}\) satisfies the independence hypothesis.

\textbf{Question 4}

a:

The sum of the other variables is multiplied by n, which means that as n
increases, the X\^{}2 statistic will also increase.

b:

Chi square tests do not return different results for ordinal data. The
test only factors in row and column totals, and not order/direction of
the data, which means that the test cannot provide information about
association.

c:

The estimated expected frequency of row i and column j:

\[{\hat{\mu}}_{ij} = \frac{n_{i+}n_{+j}}{n}\]

Total estimated expected frequency row i:,

\[=\sum_{j}\frac{n_{i+}n_{+j}}{n} = \frac{n_{i+}}{n}\sum_{j} n_{+j} = n_{i+}\]

Total estimated expected frequency of col j:

\[=\sum_{i}\frac{n_{i+}n_{+j}}{n} = \frac{n_{+j}}{n}\sum_{i} n_{i+} = n_{+j}\]

Thus, \(\{{\hat{\mu}}_{ij}\}\) have the same row and column totals as
\(\{n_{ij}\}\).

\textbf{Question 5}

\begin{Shaded}
\begin{Highlighting}[]
\NormalTok{table5 <-}\StringTok{ }\KeywordTok{matrix}\NormalTok{(}\KeywordTok{c}\NormalTok{(}\DecValTok{7}\NormalTok{,}\DecValTok{8}\NormalTok{,}\DecValTok{0}\NormalTok{,}\DecValTok{15}\NormalTok{),}\DataTypeTok{nrow =} \DecValTok{2}\NormalTok{, }\DataTypeTok{ncol =} \DecValTok{2}\NormalTok{)}
\KeywordTok{rownames}\NormalTok{(table5) <-}\StringTok{ }\KeywordTok{c}\NormalTok{(}\StringTok{"Normalized serum"}\NormalTok{, }\StringTok{"Not Normalized"}\NormalTok{)}
\KeywordTok{colnames}\NormalTok{(table5) <-}\StringTok{ }\KeywordTok{c}\NormalTok{(}\StringTok{"Treatment"}\NormalTok{, }\StringTok{"Control"}\NormalTok{)}

\KeywordTok{fisher.test}\NormalTok{(table5, }\DataTypeTok{alternative =} \StringTok{"greater"}\NormalTok{)}
\end{Highlighting}
\end{Shaded}

\begin{verbatim}
## 
##  Fisher's Exact Test for Count Data
## 
## data:  table5
## p-value = 0.003161
## alternative hypothesis: true odds ratio is greater than 1
## 95 percent confidence interval:
##  2.645931      Inf
## sample estimates:
## odds ratio 
##        Inf
\end{verbatim}

Fisher's test gives us an exact p-value unlike the chi-square test which
only gives an approximation. In this case, p = 0.003161. Since the p
value is so low, we have extremely strong evidence against our Ho that
there was no difference in results between treatment and control groups,
meaning that results were significantly better in the treatment group.

\textbf{Question 6}

a:

Let's imagine a situation in which Maine has more old people, and South
Carolina has more poor people than Maine at every age group, but less
old people. Poor people are more likely to die than rich people, leading
to South Carolina having higher death rates across all age groups. Old
people are more likely to die than young people, so in total Maine might
have a higher total death rate since they have more old than young
people.

b:

As explained by Simpson's paradox, this situation is possible:

\begin{Shaded}
\begin{Highlighting}[]
\NormalTok{table6 <-}\StringTok{ }\KeywordTok{matrix}\NormalTok{(}\KeywordTok{c}\NormalTok{(}\DecValTok{320}\NormalTok{,}\DecValTok{180}\NormalTok{,}\DecValTok{800}\NormalTok{,}\DecValTok{400}\NormalTok{,}\FloatTok{0.8}\NormalTok{,}\FloatTok{0.9}\NormalTok{,}\DecValTok{80}\NormalTok{,}\DecValTok{200}\NormalTok{,}\DecValTok{400}\NormalTok{,}\DecValTok{800}\NormalTok{,}\FloatTok{0.4}\NormalTok{,}\FloatTok{0.5}\NormalTok{,}\FloatTok{0.333}\NormalTok{,}\FloatTok{0.316}\NormalTok{), }\DataTypeTok{nrow =} \DecValTok{2}\NormalTok{, }\DataTypeTok{ncol =} \DecValTok{7}\NormalTok{)}
\KeywordTok{rownames}\NormalTok{(table6) <-}\StringTok{ }\KeywordTok{c}\NormalTok{(}\StringTok{"Jones"}\NormalTok{, }\StringTok{"Simpson"}\NormalTok{)}
\KeywordTok{colnames}\NormalTok{(table6) <-}\StringTok{ }\KeywordTok{c}\NormalTok{(}\StringTok{"Hits '19"}\NormalTok{, }\StringTok{"At Bats '19"}\NormalTok{, }\StringTok{"BA '19"}\NormalTok{, }\StringTok{"Hits '20"}\NormalTok{, }\StringTok{"At Bats '20"}\NormalTok{, }\StringTok{"BA '20"}\NormalTok{, }\StringTok{"Total BA"}\NormalTok{)}

\NormalTok{table6}
\end{Highlighting}
\end{Shaded}

\begin{verbatim}
##         Hits '19 At Bats '19 BA '19 Hits '20 At Bats '20 BA '20 Total BA
## Jones        320         800    0.8       80         400    0.4    0.333
## Simpson      180         400    0.9      200         800    0.5    0.316
\end{verbatim}

\textbf{Question 7}

a:

True. When 2 variables are independent the ratio of their odds is 1.

b:

True

c:

False. Odds ratio would be different if you used marijuana as the
denominator instead of numerator when calculating the ratio, but it has
nothing to do with if the variable is response or explanatory.

d:

True. Chi-square tests do not provide different results for ordinal data
than nominal data.

e:

False. Two variables can be conditionally independent but not generally
independent.

f:

true

\textbf{Question 8}

a:

A slope of -0.0662 in the linear probability model indicates that the
probability of the pitcher pitching a complete game is predicted to
decrease by -0.0662 for each decade that passes.

b:

0.6930 − (0.0662*12) = -0.1014. Obviously a pitcher cannot have a
negative probability of pitching a full game so this model is not valid
for x = 12.

c:

\[P(Y) = \frac{e^(a+bX)}{1+e^(a+bX)}\]

\[\hat{\pi} = \frac{exp(1.057-0.368*x)}{1+exp(1.057-0.368*x)} | x = 12, = 0.0336\]

This prediction is obviously more plausible, since it has a positive
value. A negative value is logically impossible.

\end{document}
